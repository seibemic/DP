\chapter{Conclusion}
%In this thesis we have discussed bayesian filtering and its impact on multi-target tracking systems. In real-world
%scenarios, the tracking systems have to deal with non-linear and non-Gaussian models. Moreover, the tracking systems
%must be able to handle noisy measurements and cluttered environments. To deal with such scenarios, target tracking
%systems comes with additional parameters that must be properly set according to the environment. Researchers
%investigate new methods how to determine or estimate the appropriate models simulating the targets movement patterns
%and sensor measurement uncertainties.
%
%In this work we have proposed a method to estimate one of the key parameter -- the detection probability. This
%probability plays a pivotal role in targets survivability. To estimate the detection probability of tracked objects in
%video footage for each target in
%every time step, advanced deep-learning image processing techniques are applied. Moreover, the possibility of targets
%misdetection, due to an obstacle obstructing the camera's view, is considered. To enhance the target's capability to
%survive increased period without any detection, the pruning method of GM-PHD filter is reformed.
%
%The experiments demonstrate the effectiveness of the proposed method, particularly in scenarios where tracked objects are undetectable. However, they also reveal certain weaknesses that warrant further investigation in future research endeavors.

In this thesis, we have delved into the realm of Bayesian filtering and its profound impact on multi-target tracking
systems. Real-world scenarios pose significant challenges, as tracking systems must grapple with non-linear and non-Gaussian models, as well as contend with noisy measurements and cluttered environments. Negotiating such complexities necessitates fine-tuning of additional parameters within tracking systems tailored to the specific environmental conditions. In order to eventually simulate target movement patterns and treat the problems that stem from sensor measurement uncertainties more accurately, researchers continually explore innovative methodologies to ascertain or estimate these parameters.

This work introduces an innovative approach aimed at estimating one of the key parameters crucial to tracking systems --
the detection probability. This probability is profoundly influential in the survivability of tracked objects, making
its
accurate
estimation paramount. Leveraging advanced deep-learning image processing techniques, we propose a method to estimate
the detection probability for each target at every time step within video footage. Furthermore, we address the
possibility of the target misdetection, a common occurrence when obstacles obstruct the camera's view. To endorse the
resilience of targets during extended periods without detection, we refine the pruning method of the GM-PHD filter.

The experiments demonstrate the effectiveness of the proposed method, particularly in scenarios where tracked objects are undetectable. However, they also reveal certain weaknesses that warrant further investigation in future research endeavors.




