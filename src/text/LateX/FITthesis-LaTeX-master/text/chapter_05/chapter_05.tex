\chapter{Experiments}
In order to demonstrate the efficacy of the proposed method for estimating detection probability, numerous experiments were conducted. This section meticulously examines each of these experiments, delineating the selected parameters and conducting a comparative analysis against the GM-PHD filter with static detection probability.
A mixture of video was made to examine the experiments.
\begin{itemize}
  \item \textbf{V1:} This video shot by a camera from a highway was downloaded from YouTube. The link of this video: \href{https://www.youtube.com/watch?v=KBsqQez-O4w&t=30s&ab_channel=NickMartinez}{youtube.com}
  \item \textbf{V2:} The next video comes from YouTube as well. It captures a common traffic. The link of this video: \href{https://www.youtube.com/watch?v=7WFYiZersNc&ab_channel=AbdulMunaim}{youtube.com}
  \item \textbf{V3:} To get a video of a traffic with an obstacle in the view of the camera, we recorded a video on
  our own. This video is one of them.
  \item \textbf{V4:} Another video, we recorded, with a slightly different scenario.
\end{itemize}

In all experiments as a state space model, CVM model is used, i.e., the state $x_k = [p_{x,k},p_{y,k},v_{x,k},v_{y,k}]^T$ of each target consists of two-dimensional position $(p_{x,k},p_{y,k})$ and velocity $(v_{x,k},v_{y,k})$. The measurement is the center of the mask of the detected object, which is typically noisy. The survival probability of the targets $p_{S,k} = 0.99$. A state evolution model \eqref{eq:phd_linear_model_state} is employed with
\begin{align}
  F_k &=
  \begin{bmatrix}
    I_2 & \Delta I_2 \\
    0_2 & I_2
  \end{bmatrix},
  \quad
  Q_k = \sigma_{\upsilon}^2
  \begin{bmatrix}
    \Delta I_4
  \end{bmatrix},
\end{align}
where $I_n$ and $0_n$ denote the $n\times n$ identity and zero matrices respectively and $\Delta = 1$ is the sampling
period.

For every experiment, different parameters for the model have to be applied. Parameters like $P, \sigma_{\upsilon}$
are displayed in corresponding tables.

The measurement follows the observation model \eqref{eq:phd_linear_model_measurements} with $H_k = [I_2, 0_2]$, $R_k
= \sigma_{\epsilon}^2I_2$, where $ \sigma_{\epsilon}$ is the standard deviation of the measurement noise.  $I_2$ and $0_2$ stand for
the $2\times 2$ identity and zero matrices, respectively.
The standard deviation $\sigma_{\epsilon}$ is displayed in tables as well.

Since the experiments compare the proposed method for estimating the dynamic detection probability with constant
detection probability, each table contains additional information about the necessary parameters. For constant
detection probability, it is the value of the probability. For dynamic detection probability, values of $T_H, T_C$
from Equation \eqref{eq:mphd_recursion_update_intesity_misdetect_pd}
and initial $p_{D,k}(x)$ is provided.

The pruning thresholds are the same for both GM-PHD filters, except filter with dynamic detection probability
includes second pruning threshold for pruning targets with state \textit{detected} or \textit{hidden}. The common
pruning threshold is covered in tables as $T_p$, the lowered pruning threshold as $T_l$.

The labels above
the targets are the numerical representations of the state targets: $0=\text{detected}, 1=\text{hidden}, 2=\text{dead}$.

The means of the birth places are roughly in the middle of the traffic lanes and its confidence ellipses of size of
three standard
deviations are bounded by blue circles. The red bounding boxes in figures represent bounding
boxes
given by an object detection model. The red ellipses are the covariance matrices of the targets.


\section{E1: Traffic without any obstacle}
The first experiment focuses on comparison of GM-PHD filter with constant detection probability and GM-PHD filter
with dynamic detection probability with different settings explained in Section \ref{sec:mphd_problemDef}. To analyze,
if our proposed method works on common scenarios, the video do not include any obstacle, thus the targets are
constantly visible.

The video is recorded at 29 fps. For simplicity only detections of the right side of the traffic are
taken into account in this video.

\subsection{V1 -- GM-PHD with constant detection probability}
The measurements for GM-PHD filter with constant detection probability are obtained by the YOLO object detection
model. The parameters' values are displayed in Table \ref{tab:E1-V1-S0}.
\begin{table}[!h]
  \centering
  \begin{tabular}{|c|c|c|c|c|}
    \hline
    $P_{D}$ & $P$ & $\sigma_{\upsilon}$ & $\sigma_{\epsilon}$ & $T_p$ \\ \noalign{\hrule height 1.5pt}
    0.9 & $diag(600,600,600,600)$ & 150 & 0.1 & 0.1\\
    \hline
  \end{tabular}
  \caption{The parameter settings for experiment E1-V1 with constant detection probability.}
  \label{tab:E1-V1-S0}
\end{table}


\subsection{V1 -- GM-PHD with dynamic detection probability}

\begin{table}[!h]
  \centering
  \begin{tabular}{|c|c|c|c|c|c|c|c|}
    \hline
    $P_{D,k}(x)$ & $P$ & $\sigma_{\upsilon}$ & $\sigma_{\epsilon}$ & $T_H$ & $T_d$ & $T_p$ & $T_l$\\ \noalign{\hrule
    height 1.5pt}
    0.3 & $diag(600,600,600,600)$ & 150 & 0.1 & 1 & 3 & 0.1 & 0.01\\
    \hline
  \end{tabular}
  \caption{The parameter settings for experiment E1-V1 with dynamic detection probability.}
  \label{tab:E1-V1-Sx}
\end{table}

  \subsubsection{S1 -- YOLO + YOLO}

  \subsubsection{S2 -- YOLO + SAM}

  \subsubsection{S3 -- Grounded DINO}


\subsection{V2 -- GM-PHD with constant detection probability}
\begin{table}[!h]
  \centering
  \begin{tabular}{|c|c|c|c|c|}
    \hline
    $P_{D}$ & $P$ & $\sigma_{\upsilon}$ & $\sigma_{\epsilon}$ & $T_p$ \\ \noalign{\hrule height 1.5pt}
    0.9 & $diag(600,600,600,600)$ & 150 & 0.1 & 0.1\\
    \hline
  \end{tabular}
  \caption{The parameter settings for experiment E1-V2 with constant detection probability.}
  \label{tab:E1-V2-S0}
\end{table}

\subsection{V2 -- GM-PHD with dynamic detection probability}
\begin{table}[!h]
  \centering
  \begin{tabular}{|c|c|c|c|c|c|c|c}
    \hline
    $P_{D,k}(x)$ & $P$ & $\sigma_{\upsilon}$ & $\sigma_{\epsilon}$ & $T_H$ & $T_d$ & $T_p$ & $T_l$ \\ \noalign{\hrule height 1.5pt}
    0.3 & $diag(600,600,600,600)$ & 150 & 0.1 & 1 & 3 & 0.1 & 0.01 \\
    \hline
  \end{tabular}
  \caption{The parameter settings for experiment E1-V2 with dynamic detection probability.}
  \label{tab:E1-V2-Sx}
\end{table}

\subsubsection{S1 -- YOLO + YOLO}

\subsubsection{S2 -- YOLO + SAM}

\subsubsection{S3 -- Grounded DINO}


\section{E1: paper}
\section{E2: settings comparison}
\section{E3: add obstacle, compare with PHD}
\section{E4: comparison of mean vector}
\section{E5: comparison of covariance}
