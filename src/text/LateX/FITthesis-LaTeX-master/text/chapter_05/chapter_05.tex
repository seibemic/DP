\chapter{Experiments}
In order to demonstrate the efficacy of the proposed method for estimating detection probability, numerous experiments were conducted. This section meticulously examines each of these experiments, delineating the selected parameters and conducting a comparative analysis against the GM-PHD filter with static detection probability.
A mixture of video was made to examine the experiments.
\begin{itemize}
  \item \textbf{V1:} This video shot by a camera from a highway was downloaded from YouTube. The link of this video: \href{https://www.youtube.com/watch?v=KBsqQez-O4w&t=30s&ab_channel=NickMartinez}{youtube.com}
  \item \textbf{V2:} The next video comes from YouTube as well. It captures a common traffic. The link of this video: \href{https://www.youtube.com/watch?v=7WFYiZersNc&ab_channel=AbdulMunaim}{youtube.com}
  \item \textbf{V3:} To get a video of a traffic with an obstacle in the view of the camera, we recorded a video on
  our own. This video is one of them.
  \item \textbf{V4:} Another video, we recorded, with a slightly different scenario.
\end{itemize}

In all experiments as a state space model, CVM model is used, i.e., the state $x_k = [p_{x,k},p_{y,k},v_{x,k},v_{y,k}]^T$ of each target consists of two-dimensional position $(p_{x,k},p_{y,k})$ and velocity $(v_{x,k},v_{y,k})$. The measurement is the center of the mask of the detected object, which is typically noisy. The survival probability of the targets $p_{S,k} = 0.99$. A state evolution model \eqref{eq:phd_linear_model_state} is employed with
\begin{align}
  F_k &=
  \begin{bmatrix}
    I_2 & \Delta I_2 \\
    0_2 & I_2
  \end{bmatrix},
  \quad
  Q_k = \sigma_{\upsilon}^2
  \begin{bmatrix}
    \Delta I_4
  \end{bmatrix},
\end{align}
where $I_n$ and $0_n$ denote the $n\times n$ identity and zero matrices respectively and $\Delta = 1$ is the sampling
period.

For every experiment, different parameters for the model have to be applied. Parameters like $P, \sigma_{\upsilon}$
are displayed in corresponding tables.

The measurement follows the observation model \eqref{eq:phd_linear_model_measurements} with $H_k = [I_2, 0_2]$, $R_k
= \sigma_{\epsilon}^2I_2$, where $ \sigma_{\epsilon}$ is the standard deviation of the measurement noise.  $I_2$ and $0_2$ stand for
the $2\times 2$ identity and zero matrices, respectively.
The standard deviation $\sigma_{\epsilon}$ is displayed in tables as well.

Since the experiments compare the proposed method for estimating the dynamic detection probability with constant
detection probability, each table contains additional information about the necessary parameters. For constant
detection probability, it is the value of the probability. For dynamic detection probability, values of $T_H, T_C$
from Equation \eqref{eq:mphd_recursion_update_intesity_misdetect_pd}
and initial $p_{D,k}(x)$ is provided.

The pruning thresholds are the same for both GM-PHD filters, except filter with dynamic detection probability
includes second pruning threshold for pruning targets with state \textit{detected} or \textit{hidden}. The basic
pruning threshold is covered in tables as $T_p$, the lowered pruning threshold as $T_l$.

In addition, object detectors' results are dependent on thresholds as well. The Yolo model requires confidence threshold $T_{YOLO}$ for objects visualization. The Grounding DINO model requires two thresholds as it detects objects based on text input. The text input threshold $T_{text}$ and bounding box threshold $T_{bbox}$ for given experiment are included in corresponding parameter settings tables.

The labels above
the targets are the numerical representations of the state targets: $0=\text{detected}, 1=\text{hidden}, 2=\text{dead}$.

The means of the birth places are roughly in the middle of the traffic lanes and its confidence ellipses of size of
three standard
deviations are bounded by blue circles. The red bounding boxes in figures represent bounding
boxes
given by an object detection model. The red ellipses are the covariance matrices of the targets.


\section{E1: Traffic without any obstacle}
The first experiment focuses on comparison of GM-PHD filter with constant detection probability and GM-PHD filter
with dynamic detection probability with different settings explained in Section \ref{sec:mphd_problemDef}. To analyze,
if our proposed method works on common scenarios, videos do not include any obstacle, thus the targets are
constantly visible.



\subsection{V1}
The video is recorded at 29 fps. For simplicity only detections of the right side of the traffic are
taken into account in this video. In this experiments frames 36-79 of video V1 are shown, as it includes some curios situations.
\subsection{V1 -- GM-PHD with constant detection probability}
The measurements for GM-PHD filter with constant detection probability are obtained by the YOLO object detection
model. The parameters' values are displayed in Table \ref{tab:E1-V1-S0}.
\begin{table}[!h]
    \centering
    \begin{tabular}{|c|c|c|c|c|c|}
        \hline
        $P_{D}$ & $P$ & $\sigma_{\upsilon}$ & $\sigma_{\epsilon}$ & $T_p$ & $T_{YOLO}$ \\ \noalign{\hrule height 1.5pt}
        0.9 & $diag(600,600,600,600)$ & 0.1 & 150 & 0.1 & 0.3\\
        \hline
    \end{tabular}
    \caption{The parameter settings for experiment E1-V1 with constant detection probability.}
    \label{tab:E1-V1-S0}
\end{table}

Figure \ref{fig:E1-V1-S0} shows some highlights of the GM-PHD filter with the constant detection probability.
\begin{itemize}
    \item \textbf{\ref{fig:E1-V1-S0:01}:} This is the starting frame, where four cars were previously detected, thus they became our observed targets. In the distance we see that one more car was detected by YOLO, but it did not cross any spawning point, so it does not count into observed targets.
    \item \textbf{\ref{fig:E1-V1-S0:02}:} Another car is approaching to the scene and should be initialized soon. The YOLO model is not able to always detect all desired objects, as it happened here with the left distanced car. As the detection probality is 0.9, this target did not survive the pruning step and is lost now.
    \item \textbf{\ref{fig:E1-V1-S0:03}:} The previously lost car is detected again, but does not belong to the set of observed targets furthermore.
    \item \textbf{\ref{fig:E1-V1-S0:04}:} The new car crossed the spawning point, but the YOLO model was not able to detect this kind of car for many frames in a row, thus this car is not tracked at all.
    \item \textbf{\ref{fig:E1-V1-S0:05}:} Two cars on the right were not detected again. But this time onw of the targets was able to survive, so we continue to track at least one of the cars.
    \item \textbf{\ref{fig:E1-V1-S0:06}:} The previously undetected cars are detected again and both cars are tracked again.
    \item \textbf{\ref{fig:E1-V1-S0:07}:} Another car arrives to the scene and this time it is successfully detected and initialized.
    \item \textbf{\ref{fig:E1-V1-S0:08}:} Even though all six cars are detected by YOLO, only three targets appear in the scene. One of the targets covers two cars at once so we can guess that four out of six targets are covered by the GM-PHD filter.
\end{itemize}

In Graph \ref{gr:E1-V1-S0} it is clearly seen that even though the number of targets is increasing, the misdetection of the YOLO model causes the targets loss. The "All targets in set" line is fully covered by the blue line, thus the targets are absolutely lost, thus they can not be reborn by a measurement.

This experiment shows us that the GM-PHD filter is able to accurately track the position of objects. With detection probablity $p_D = 0.9$ and without modified pruning technique, the filter is very sensitive to the output of object detector and if YOLO is not able to detect all desired objects, targets could be easily lost.

\begin{figure}[H]
    \centering
    \includegraphics[width=\linewidth]{../../../experiments/E1/V1/noPd/staticPd_det}
    \caption{Development chart of number of detected targets, targets in filter's queue, displayed targets and true targets' count.}
    \label{gr:E1-V1-S0}
\end{figure}

\begin{figure}[H]
    \centering
    \begin{subfigure}{0.48\textwidth}
        \centering
        \includegraphics[width=\linewidth]{../../../experiments/E1/V1/noPd/36}
        \caption{Frame number: 36.}
        \label{fig:E1-V1-S0:01}
    \end{subfigure}
    \begin{subfigure}{0.48\textwidth}
        \centering
        \includegraphics[width=\linewidth]{../../../experiments/E1/V1/noPd/48}
        \caption{Frame number: 48.}
        \label{fig:E1-V1-S0:02}
    \end{subfigure}
    \\
    \begin{subfigure}{0.48\textwidth}
        \centering
        \includegraphics[width=\linewidth]{../../../experiments/E1/V1/noPd/49}
        \caption{Frame number: 49.}
        \label{fig:E1-V1-S0:03}
    \end{subfigure}
    \begin{subfigure}{0.48\textwidth}
        \centering
        \includegraphics[width=\linewidth]{../../../experiments/E1/V1/noPd/56}
        \caption{Frame number: 56.}
        \label{fig:E1-V1-S0:04}
    \end{subfigure}
    \\
    \begin{subfigure}{0.48\textwidth}
        \centering
        \includegraphics[width=\linewidth]{../../../experiments/E1/V1/noPd/67}
        \caption{Frame number: 67.}
        \label{fig:E1-V1-S0:05}
    \end{subfigure}
    \begin{subfigure}{0.48\textwidth}
        \centering
        \includegraphics[width=\linewidth]{../../../experiments/E1/V1/noPd/69}
        \caption{Frame number: 69.}
        \label{fig:E1-V1-S0:06}
    \end{subfigure}
    \\
    \begin{subfigure}{0.48\textwidth}
        \centering
        \includegraphics[width=\linewidth]{../../../experiments/E1/V1/noPd/73}
        \caption{Frame number: 73.}
        \label{fig:E1-V1-S0:07}
    \end{subfigure}
    \begin{subfigure}{0.48\textwidth}
        \centering
        \includegraphics[width=\linewidth]{../../../experiments/E1/V1/noPd/79}
        \caption{Frame number: 79.}
        \label{fig:E1-V1-S0:08}
    \end{subfigure}
    \caption{Image sequence of tracked objects using GM-PHD filter with constant detection probability.}
    \label{fig:E1-V1-S0}
\end{figure}

\subsection{V1 -- GM-PHD with dynamic detection probability}
Following experiments test GM-PHD filter with the dynamic detection probability and the modified pruning on the video \textit{V1}.
\subsubsection{S1 -- YOLO + YOLO}
This experiment uses settings \textit{S1}, i.e, the YOLO model provides with both object detection bboxes and segmentation masks.
The parameter settings are shown in Table \ref{tab:E1-V1-S1}.
\begin{table}[H]
    \centering
    \begin{tabular}{|c|c|c|c|c|c|c|c|c|}
        \hline
        $P_{D,k}(x)$ & $P$ & $\sigma_{\upsilon}$ & $\sigma_{\epsilon}$ & $T_H$ & $T_d$ & $T_p$ & $T_l$ & $T_{YOLO}$ \\ \noalign{\hrule
        height 1.5pt}
        0.3 & $diag(600,600,600,600)$ & 0.1 & 150 & 1 & 3 & 0.1 & 0.01 & 0.3\\
        \hline
    \end{tabular}
    \caption{The parameter settings for experiment E1-V1-S1 with dynamic detection probability.}
    \label{tab:E1-V1-S1}
\end{table}

Figure \ref{fig:E1-V1-S1} shows the performance of the GM-PHD filter with the dynamic detection probability with settings \textit{S1}.
\begin{itemize}
    \item \textbf{\ref{fig:E1-V1-S1:01}:} As in previous experiment the first frame starts with four cars that were previously detected. The car that is far away is detected, but not initialized, as it have not crossed any spawning point.
    \item \textbf{\ref{fig:E1-V1-S1:02}:} In the frame 48, the left distanced car was not detected as in previous experiment. But due to the high detection probability and modified pruning, the target is able to survive.
    \item \textbf{\ref{fig:E1-V1-S1:03}:} The previously undetected car is detected again and the target survives even with misdetection in previous frames.
    \item \textbf{\ref{fig:E1-V1-S1:04}:} The new car crossed the spawning point, but the YOLO model is not able to detect this vehicle for many consecutive frames, thus this car is not initialized.
    \item \textbf{\ref{fig:E1-V1-S1:05}:} Two cars on the right are not detected again but targets survive.
    \item \textbf{\ref{fig:E1-V1-S1:06}:} The previously undetected cars are detected again and both targets get their measurements increasing their weight.
    \item \textbf{\ref{fig:E1-V1-S1:07}:} Another car arrives to the scene, and it is successfully initialized.
    \item \textbf{\ref{fig:E1-V1-S1:08}:} The undetected car on the left caught up other targets. As a result of this is initializing this target, even though it is misdetected most of the time. And so we have six true targets and also six tracked targets.
\end{itemize}

The graph \ref{gr:E1-V1-S1} presents better performance of the GM-PHD filter. Even though not all the targets are displayed when they should, most of the time, there are more targets is a queue that do not have weight high enough to be displayed. As soon as these targets get their measurements, their weight increases and are back on the scene.

Not only the number of displayed targets is closer to the true targets count, but also targets waiting in the queue exceeds the true count, so we are aware of the potential targets, that can appear in the scene.

\begin{figure}[H]
    \centering
    \includegraphics[width=\linewidth]{../../../experiments/E1/V1/YOLO/yolo_det}
    \caption{Development chart of number of detected targets, targets in filter's queue, displayed targets and true targets' count.}
    \label{gr:E1-V1-S1}
\end{figure}

\begin{figure}[H]
    \centering
    \begin{subfigure}{0.48\textwidth}
        \centering
        \includegraphics[width=\linewidth]{../../../experiments/E1/V1/YOLO/36}
        \caption{Frame number: 36.}
        \label{fig:E1-V1-S1:01}
    \end{subfigure}
    \begin{subfigure}{0.48\textwidth}
        \centering
        \includegraphics[width=\linewidth]{../../../experiments/E1/V1/YOLO/48}
        \caption{Frame number: 48.}
        \label{fig:E1-V1-S1:02}
    \end{subfigure}
    \\
    \begin{subfigure}{0.48\textwidth}
        \centering
        \includegraphics[width=\linewidth]{../../../experiments/E1/V1/YOLO/50}
        \caption{Frame number: 50.}
        \label{fig:E1-V1-S1:03}
    \end{subfigure}
    \begin{subfigure}{0.48\textwidth}
        \centering
        \includegraphics[width=\linewidth]{../../../experiments/E1/V1/YOLO/56}
        \caption{Frame number: 56.}
        \label{fig:E1-V1-S1:04}
    \end{subfigure}
    \\
    \begin{subfigure}{0.48\textwidth}
        \centering
        \includegraphics[width=\linewidth]{../../../experiments/E1/V1/YOLO/67}
        \caption{Frame number: 67.}
        \label{fig:E1-V1-S1:05}
    \end{subfigure}
    \begin{subfigure}{0.48\textwidth}
        \centering
        \includegraphics[width=\linewidth]{../../../experiments/E1/V1/YOLO/69}
        \caption{Frame number: 69.}
        \label{fig:E1-V1-S1:06}
    \end{subfigure}
    \\
    \begin{subfigure}{0.48\textwidth}
        \centering
        \includegraphics[width=\linewidth]{../../../experiments/E1/V1/YOLO/73}
        \caption{Frame number: 73.}
        \label{fig:E1-V1-S1:07}
    \end{subfigure}
    \begin{subfigure}{0.48\textwidth}
        \centering
        \includegraphics[width=\linewidth]{../../../experiments/E1/V1/YOLO/79}
        \caption{Frame number: 79.}
        \label{fig:E1-V1-S1:08}
    \end{subfigure}
    \caption{Image sequence of tracked objects using GM-PHD filter with dynamic detection probability and YOLO only.}
    \label{fig:E1-V1-S1}
\end{figure}







\subsubsection{S2 -- YOLO + SAM}
This experiment employs configuration \textit{S2}, wherein the YOLO model furnishes objects' bounding boxes and the SAM model furnishes segmentation masks.
The parameter configurations can be found in Table \ref{tab:E1-V1-S2}.
\begin{table}[H]
    \centering
    \begin{tabular}{|c|c|c|c|c|c|c|c|c|}
        \hline
        $P_{D,k}(x)$ & $P$ & $\sigma_{\upsilon}$ & $\sigma_{\epsilon}$ & $T_H$ & $T_d$ & $T_p$ & $T_l$ & $T_{YOLO}$ \\ \noalign{\hrule
        height 1.5pt}
        0.3 & $diag(600,600,600,600)$ & 0.1 & 150 & 1 & 3 & 0.1 & 0.01 & 0.3\\
        \hline
    \end{tabular}
    \caption{The parameter settings for experiment E1-V1-S2 with dynamic detection probability.}
    \label{tab:E1-V1-S2}
\end{table}

Figure \ref{fig:E1-V1-S2} illustrates the GM-PHD filter's performance with dynamic detection probability, employing settings \textit{S2}.
This sequence is very similar to the previous experiment. There are four targets at the beginning and they are tracked successfully the whole time. At frame \ref{fig:E1-V1-S2:06} two of the targets were not detected, but both survived. The YOLO model is not able to detect the fifth car, but it is initialized lated due to the other target.

The graph \ref{gr:E1-V1-S2} shows a better stability of number of tracked targets. This might be cased by the fact, that object detection YOLO model gives slightly different results than the object detection YOLO model with segmentation. The "All targets in set" orange line representing the number of targets in filter's queue is also more accurate to the true count, deflecting only be one target at maximum.

Settings \textit{S2} perform a little better than settings \textit{S1}. The number of tracked objects is closer to the true value.

\begin{figure}[H]
    \centering
    \includegraphics[width=\linewidth]{../../../experiments/E1/V1/SAM/sam_det}
    \caption{Development chart of number of detected targets, targets in filter's queue, displayed targets and true targets' count.}
    \label{gr:E1-V1-S2}
\end{figure}

\begin{figure}[H]
    \centering
    \begin{subfigure}{0.48\textwidth}
        \centering
        \includegraphics[width=\linewidth]{../../../experiments/E1/V1/SAM/36}
        \caption{Frame number: 36.}
        \label{fig:E1-V1-S2:01}
    \end{subfigure}
    \begin{subfigure}{0.48\textwidth}
        \centering
        \includegraphics[width=\linewidth]{../../../experiments/E1/V1/SAM/48}
        \caption{Frame number: 48.}
        \label{fig:E1-V1-S2:02}
    \end{subfigure}
    \\
    \begin{subfigure}{0.48\textwidth}
        \centering
        \includegraphics[width=\linewidth]{../../../experiments/E1/V1/SAM/53}
        \caption{Frame number: 53.}
        \label{fig:E1-V1-S2:03}
    \end{subfigure}
    \begin{subfigure}{0.48\textwidth}
        \centering
        \includegraphics[width=\linewidth]{../../../experiments/E1/V1/SAM/57}
        \caption{Frame number: 57.}
        \label{fig:E1-V1-S2:04}
    \end{subfigure}
    \\
    \begin{subfigure}{0.48\textwidth}
        \centering
        \includegraphics[width=\linewidth]{../../../experiments/E1/V1/SAM/62}
        \caption{Frame number: 62.}
        \label{fig:E1-V1-S2:05}
    \end{subfigure}
    \begin{subfigure}{0.48\textwidth}
        \centering
        \includegraphics[width=\linewidth]{../../../experiments/E1/V1/SAM/69}
        \caption{Frame number: 69.}
        \label{fig:E1-V1-S2:06}
    \end{subfigure}
    \\
    \begin{subfigure}{0.48\textwidth}
        \centering
        \includegraphics[width=\linewidth]{../../../experiments/E1/V1/SAM/70}
        \caption{Frame number: 70.}
        \label{fig:E1-V1-S2:07}
    \end{subfigure}
    \begin{subfigure}{0.48\textwidth}
        \centering
        \includegraphics[width=\linewidth]{../../../experiments/E1/V1/SAM/78}
        \caption{Frame number: 78.}
        \label{fig:E1-V1-S2:08}
    \end{subfigure}
    \caption{Image sequence of tracked objects using GM-PHD filter with dynamic detection probability, the YOLO object detector and the SAM image segmentation model.}
    \label{fig:E1-V1-S2}
\end{figure}


\subsubsection{S3 -- Grounded DINO}
The experiment with settings \textit{S3} uses Grounding DINO object detector and the SAM image segmentation model.
All used parameters are included in Table \ref{tab:E1-V1-S3}.
\begin{table}[H]
    \centering
    \begin{tabular}{|c|c|c|c|c|c|c|c|c|c|}
        \hline
        $P_{D,k}(x)$ & $P$ & $\sigma_{\upsilon}$ & $\sigma_{\epsilon}$ & $T_H$ & $T_d$ & $T_p$ & $T_l$ & $T_{text}$ & $T_{bbox}$\\ \noalign{\hrule
        height 1.5pt}
        0.3 & $diag(600,600,600,600)$ & 0.1 & 150 & 1 & 3 & 0.1 & 0.01 & 0.3 & 0.3\\
        \hline
    \end{tabular}
    \caption{The parameter settings for experiment E1-V1-S3 with dynamic detection probability.}
    \label{tab:E1-V1-S3}
\end{table}

Figure \ref{fig:E1-V1-S3} shows the performance of the GM-PHD filter with the dynamic detection probability with settings \textit{S3}.
\begin{itemize}
    \item \textbf{\ref{fig:E1-V1-S3:01}:} Due to the different object detection model, there are more cars detected in the scene. Only four cars drove through the spawn points till this moment, so only there objects are counted to the true count.
    \item \textbf{\ref{fig:E1-V1-S3:02}:} All targets are tracked successfully.
    \item \textbf{\ref{fig:E1-V1-S3:03}:} Unlike YOLO, Grounding DINO is able to detect the arriving car.
    \item \textbf{\ref{fig:E1-V1-S3:07}:} The next arriving car is detected and initialized as well.
    \item \textbf{\ref{fig:E1-V1-S3:08}:} In this frame new problems arises. As cars are far away, closer to each other and the model is still able to detect all cars, we gain too many new targets. For this model, different observation noise is neccessary. Moreover, for these kind of situations, dynamic model and observation noise would be appropriate ssolution.
\end{itemize}

In graph \ref{gr:E1-V1-S3} we see, that the number of detected objects is far beyond the true count. But the number of displayed targets is almost perfect till frame number 66. After that as targets are closer to each other, the number of targets grows rapidly, causing errors to the number of tracked objects.

This settings outperforms the other settings by far. The great performance of Grounding DINO causes another problems arising from characteristics of the video. The video is taken from an angle, which makes cars smaller as they go. This dynamics do go together well with static measurement and observation noise.

\begin{figure}[H]
    \centering
    \includegraphics[width=\linewidth]{../../../experiments/E1/V1/DINO/dino_det}
    \caption{Development chart of number of detected targets, targets in filter's queue, displayed targets and true targets' count.}
    \label{gr:E1-V1-S3}
\end{figure}

\begin{figure}[H]
    \centering
    \begin{subfigure}{0.48\textwidth}
        \centering
        \includegraphics[width=\linewidth]{../../../experiments/E1/V1/DINO/36}
        \caption{Frame number: 36.}
        \label{fig:E1-V1-S3:01}
    \end{subfigure}
    \begin{subfigure}{0.48\textwidth}
        \centering
        \includegraphics[width=\linewidth]{../../../experiments/E1/V1/DINO/48}
        \caption{Frame number: 48.}
        \label{fig:E1-V1-S3:02}
    \end{subfigure}
    \\
    \begin{subfigure}{0.48\textwidth}
        \centering
        \includegraphics[width=\linewidth]{../../../experiments/E1/V1/DINO/54}
        \caption{Frame number: 54.}
        \label{fig:E1-V1-S3:03}
    \end{subfigure}
    \begin{subfigure}{0.48\textwidth}
        \centering
        \includegraphics[width=\linewidth]{../../../experiments/E1/V1/DINO/58}
        \caption{Frame number: 58.}
        \label{fig:E1-V1-S3:04}
    \end{subfigure}
    \\
    \begin{subfigure}{0.48\textwidth}
        \centering
        \includegraphics[width=\linewidth]{../../../experiments/E1/V1/DINO/67}
        \caption{Frame number: 67.}
        \label{fig:E1-V1-S3:05}
    \end{subfigure}
    \begin{subfigure}{0.48\textwidth}
        \centering
        \includegraphics[width=\linewidth]{../../../experiments/E1/V1/DINO/69}
        \caption{Frame number: 69.}
        \label{fig:E1-V1-S3:06}
    \end{subfigure}
    \\
    \begin{subfigure}{0.48\textwidth}
        \centering
        \includegraphics[width=\linewidth]{../../../experiments/E1/V1/DINO/73}
        \caption{Frame number: 73.}
        \label{fig:E1-V1-S3:07}
    \end{subfigure}
    \begin{subfigure}{0.48\textwidth}
        \centering
        \includegraphics[width=\linewidth]{../../../experiments/E1/V1/DINO/78}
        \caption{Frame number: 79.}
        \label{fig:E1-V1-S3:08}
    \end{subfigure}
    \caption{Image sequence of tracked objects using the GM-PHD filter with the dynamic detection probability and the Grounded DINO model.}
    \label{fig:E1-V1-S3}
\end{figure}

\subsection{V2}
Video \textit{V2} is recorded at 29 fps. Only the cars driving from the left to the right are detected and tracked. In
the following
experiments
frames 83-109 of the
video \textit{V2} are
analyzed.
\subsection{V2 -- GM-PHD with the constant detection probability}
The measurements for the GM-PHD filter with the constant detection probability are obtained by the YOLO object detection
model. The parameters' values are displayed in Table \ref{tab:E1-V2-S0}.
\begin{table}[!h]
    \centering
    \begin{tabular}{|c|c|c|c|c|c|}
        \hline
        $P_{D}$ & $P$ & $\sigma_{\upsilon}$ & $\sigma_{\epsilon}$ & $T_p$ & $T_{YOLO}$ \\ \noalign{\hrule height 1.5pt}
        0.9 & $\diag(100,100,100,100)$ & 0.1 & 30 & 0.1 & 0.3\\
        \hline
    \end{tabular}
    \caption{The parameter settings for Experiment E1-V2 with the constant detection probability.}
    \label{tab:E1-V2-S0}
\end{table}

Figure \ref{fig:E1-V2-S0} displays the performance of the GM-PHD filter with the constant detection probability.
\begin{itemize}
    \item \textbf{\ref{fig:E1-V2-S0:01}:} The sequence begins with the frame number 83, wherein four targets have
    already been initialized and successfully detected.
    \item \textbf{\ref{fig:E1-V2-S0:02}:} Notably, the objects appear relatively small in comparison to the overall
    frame size. This size discrepancy contributes to a phenomenon where targets in a close proximity share
    measurements, resulting in the appearance of additional targets in the scene.
    \item \textbf{\ref{fig:E1-V2-S0:03}:} Despite the YOLO model failing to detect the fourth car, the target persists within the tracking system.
    \item \textbf{\ref{fig:E1-V2-S0:04}:} The previously undetected car is successfully identified once more, and
    the tracking of the target continues. Additionally, another car approaches the spawning point.
    \item \textbf{\ref{fig:E1-V2-S0:05}:} Regrettably, the car within the spawning area remains undetected and
    has not survived, probably
    due to its weight being insufficient to ensure survival. Concurrently, the first car exits the scene.
    \item \textbf{\ref{fig:E1-V2-S0:06}:} Subsequently, the car at the spawning point is detected once again and remains sufficiently close to be initialized as a target. The scene concludes with four true objects and four accurately tracked targets.
\end{itemize}


The GM-PHD filter with the constant detection probability is accurate in scenarios, where the object detector does not
miss detections regularly. Figure \ref{gr:E1-V2-S0} shows that the number of displayed targets is close enough
to the true count. Even false detections created by YOLO did not mislead the filter.


\begin{figure}[H]
    \centering
    \includegraphics[width=\linewidth]{../../../experiments/E1/V2/noPd/staticPd_det}
    \caption{Development chart of the number of detected targets, targets in the filter's queue, displayed targets
    and the
    true
    targets' count.}
    \label{gr:E1-V2-S0}
\end{figure}

\begin{figure}[H]
    \centering
    \begin{subfigure}{0.48\textwidth}
        \centering
        \includegraphics[width=\linewidth]{../../../experiments/E1/V2/noPd/83}
        \caption{Frame number: 83.}
        \label{fig:E1-V2-S0:01}
    \end{subfigure}
    \begin{subfigure}{0.48\textwidth}
        \centering
        \includegraphics[width=\linewidth]{../../../experiments/E1/V2/noPd/90}
        \caption{Frame number: 90.}
        \label{fig:E1-V2-S0:02}
    \end{subfigure}
    \\
    \begin{subfigure}{0.48\textwidth}
        \centering
        \includegraphics[width=\linewidth]{../../../experiments/E1/V2/noPd/95}
        \caption{Frame number: 95.}
        \label{fig:E1-V2-S0:03}
    \end{subfigure}
    \begin{subfigure}{0.48\textwidth}
        \centering
        \includegraphics[width=\linewidth]{../../../experiments/E1/V2/noPd/102}
        \caption{Frame number: 102.}
        \label{fig:E1-V2-S0:04}
    \end{subfigure}
    \\
    \begin{subfigure}{0.48\textwidth}
        \centering
        \includegraphics[width=\linewidth]{../../../experiments/E1/V2/noPd/105}
        \caption{Frame number: 105.}
        \label{fig:E1-V2-S0:05}
    \end{subfigure}
    \begin{subfigure}{0.48\textwidth}
        \centering
        \includegraphics[width=\linewidth]{../../../experiments/E1/V2/noPd/110}
        \caption{Frame number: 110.}
        \label{fig:E1-V2-S0:06}
    \end{subfigure}
    \caption{Image sequence of tracked objects using the GM-PHD filter with the constant detection probability.}
    \label{fig:E1-V2-S0}
\end{figure}


\subsection{V2 -- GM-PHD with the dynamic detection probability}
Experiments carried out on the video \textit{V2} using the GM-PHD filter with the dynamic detection probability and
different
settings
are
demonstrated in following sections.
\subsubsection{S1 -- YOLO + YOLO}
This experiment uses settings \textit{S1}, where the YOLO model provides both object detection bboxes and
segmentation masks.
The parameter settings are shown in Table \ref{tab:E1-V2-S1}.
\begin{table}[H]
    \centering
    \begin{tabular}{|c|c|c|c|c|c|c|c|c|}
        \hline
        $P_{D,k}(x)$ & $P$ & $\sigma_{\upsilon}$ & $\sigma_{\epsilon}$ & $T_H$ & $T_d$ & $T_p$ & $T_l$ & $T_{YOLO}$ \\ \noalign{\hrule
        height 1.5pt}
        0.3 & $\diag(600,600,600,600)$ & 0.1 & 30 & 1 & 3 & 0.1 & 0.01 & 0.3\\
        \hline
    \end{tabular}
    \caption{The parameter settings for Experiment E1-V2-S1 with the dynamic detection probability.}
    \label{tab:E1-V2-S1}
\end{table}

Figure \ref{fig:E1-V2-S1} shows the performance of the GM-PHD filter with the dynamic detection probability with settings \textit{S1}.
\begin{itemize}
    \item \textbf{\ref{fig:E1-V2-S1:01}:} Analogously to the previous analysis, four targets have surpassed the
    spawning point and are tracked by the GM-PHD filter. The targets' presence in the close proximity results in an
    additional false targets' existence.
    \item \textbf{\ref{fig:E1-V2-S1:02}:} The problem of targets' neighboring persists.
    \item \textbf{\ref{fig:E1-V2-S1:03}:} Moreover, the YOLO model classifies the targets' shadows as another cars,
    causing the initialization of extra targets.
    \item \textbf{\ref{fig:E1-V2-S1:04}:} A new car crosses the spawning point, yet the YOLO model fails to detect it.
    \item \textbf{\ref{fig:E1-V2-S1:05}:} The new car has been previously detected and
    initialized. The two
    neighbouring cars still generate additional false targets.
    \item \textbf{\ref{fig:E1-V2-S1:06}:} Finally, there appear only two targets representing the two cars. The other
    two
    cars are tracked properly.
\end{itemize}

Figure \ref{gr:E1-V2-S1} depicts a similar performance of the GM-PHD filter with the dynamic detection probability as
the GM-PHD filter with the constant detection probability. The number of displayed targets exceeds the true number of
targets due to already presented reasons.


The improved ability of the target's survival resulted in a slightly decreased the overall tracking performance. The
YOLO model
has been able to detect tracked objects flawlessly, which leads to needlessness of the dynamic detection
probability
and modified pruning method enabling enhanced targets' ability to survive.

However, the goal of this experiment is to examine the tracking capability of the GM-PHD filter with the proposed
dynamic
detection probability in common flawless scenarios. This experiment verifies the method to be sufficient.

\begin{figure}[H]
    \centering
    \includegraphics[width=\linewidth]{../../../experiments/E1/V2/YOLO/yolo_det}
    \caption{Development chart of the number of detected targets, targets in the filter's queue, displayed targets
    and the
    true targets' count.}
    \label{gr:E1-V2-S1}
\end{figure}

\begin{figure}[H]
    \centering
    \begin{subfigure}{0.48\textwidth}
        \centering
        \includegraphics[width=\linewidth]{../../../experiments/E1/V2/YOLO/83}
        \caption{Frame number: 83.}
        \label{fig:E1-V2-S1:01}
    \end{subfigure}
    \begin{subfigure}{0.48\textwidth}
        \centering
        \includegraphics[width=\linewidth]{../../../experiments/E1/V2/YOLO/89}
        \caption{Frame number: 89.}
        \label{fig:E1-V2-S1:02}
    \end{subfigure}
    \\
    \begin{subfigure}{0.48\textwidth}
        \centering
        \includegraphics[width=\linewidth]{../../../experiments/E1/V2/YOLO/94}
        \caption{Frame number: 94.}
        \label{fig:E1-V2-S1:03}
    \end{subfigure}
    \begin{subfigure}{0.48\textwidth}
        \centering
        \includegraphics[width=\linewidth]{../../../experiments/E1/V2/YOLO/98}
        \caption{Frame number: 98.}
        \label{fig:E1-V2-S1:04}
    \end{subfigure}
    \\
    \begin{subfigure}{0.48\textwidth}
        \centering
        \includegraphics[width=\linewidth]{../../../experiments/E1/V2/YOLO/103}
        \caption{Frame number: 103.}
        \label{fig:E1-V2-S1:05}
    \end{subfigure}
    \begin{subfigure}{0.48\textwidth}
        \centering
        \includegraphics[width=\linewidth]{../../../experiments/E1/V2/YOLO/109}
        \caption{Frame number: 109.}
        \label{fig:E1-V2-S1:06}
    \end{subfigure}
    \caption{Image sequence of tracked objects using the GM-PHD filter with the dynamic detection probability and YOLO
    only.}
    \label{fig:E1-V2-S1}
\end{figure}





\subsubsection{S2 -- YOLO + SAM}
As in experiment \textit{E1}, the next settings employs \textit{S2} with the YOLO object detector and the SAM
segmentation model.
All parameters are included in Table \ref{tab:E1-V2-S2}.
\begin{table}[H]
    \centering
    \begin{tabular}{|c|c|c|c|c|c|c|c|c|}
        \hline
        $P_{D,k}(x)$ & $P$ & $\sigma_{\upsilon}$ & $\sigma_{\epsilon}$ & $T_H$ & $T_d$ & $T_p$ & $T_l$ & $T_{YOLO}$ \\ \noalign{\hrule
        height 1.5pt}
        0.3 & $\diag(100,100,100,100)$ & 0.1 & 30 & 1 & 3 & 0.1 & 0.01 & 0.3\\
        \hline
    \end{tabular}
    \caption{The parameter settings for Experiment E1-V2-S2 with the dynamic detection probability.}
    \label{tab:E1-V2-S2}
\end{table}


The situation resembles the situation with settings \textit{S1}. Four targets occur in Figure \ref{fig:E1-V2-S2:01}.
They continue in their path while all targets are tracked successfully with one exception. No additional targets appear
in the subsequent frames. The YOLO model detects a car's shadow and clasifies it as an another car, which makes
the added target in Figure \ref{fig:E1-V2-S2:03}. Due to the merging step, this false target does not survive to the subsequent
frames. A new car is initialized in Figure \ref{fig:E1-V2-S2:04}. All targets are tracked in the
last frames.

The number of displayed targets is close to the true number of targets in Figure
\ref{gr:E1-V2-S2}. The number of
detected targets exceeds the number of true targets for majority of the time, due to the presence of false
detections.

Settings \textit{S2} exhibit a slight improvement over settings \textit{S1}, which is reflected in the reduced error
between the number of displayed targets and the true count.

\begin{figure}[H]
    \centering
    \includegraphics[width=\linewidth]{../../../experiments/E1/V2/SAM/sam_det}
    \caption{Development chart of the number of detected targets, targets in the filter's queue, displayed targets and
    true targets' count.}
    \label{gr:E1-V2-S2}
\end{figure}

\begin{figure}[H]
    \centering
    \begin{subfigure}{0.48\textwidth}
        \centering
        \includegraphics[width=\linewidth]{../../../experiments/E1/V2/SAM/83}
        \caption{Frame number: 83.}
        \label{fig:E1-V2-S2:01}
    \end{subfigure}
    \begin{subfigure}{0.48\textwidth}
        \centering
        \includegraphics[width=\linewidth]{../../../experiments/E1/V2/SAM/89}
        \caption{Frame number: 89.}
        \label{fig:E1-V2-S2:02}
    \end{subfigure}
    \\
    \begin{subfigure}{0.48\textwidth}
        \centering
        \includegraphics[width=\linewidth]{../../../experiments/E1/V2/SAM/94}
        \caption{Frame number: 94.}
        \label{fig:E1-V2-S2:03}
    \end{subfigure}
    \begin{subfigure}{0.48\textwidth}
        \centering
        \includegraphics[width=\linewidth]{../../../experiments/E1/V2/SAM/98}
        \caption{Frame number: 98.}
        \label{fig:E1-V2-S2:04}
    \end{subfigure}
    \\
    \begin{subfigure}{0.48\textwidth}
        \centering
        \includegraphics[width=\linewidth]{../../../experiments/E1/V2/SAM/103}
        \caption{Frame number: 103.}
        \label{fig:E1-V2-S2:05}
    \end{subfigure}
    \begin{subfigure}{0.48\textwidth}
        \centering
        \includegraphics[width=\linewidth]{../../../experiments/E1/V2/SAM/109}
        \caption{Frame number: 109.}
        \label{fig:E1-V2-S2:06}
    \end{subfigure}
    \caption{Image sequence of tracked objects using the GM-PHD filter with the dynamic detection probability, the YOLO
    object detector and the SAM image segmentation model.}
    \label{fig:E1-V2-S2}
\end{figure}


\subsubsection{S3 -- Grounded SAM}
The combination of Grounding DINO and SAM is tested using the video \textit{V2} as well. Used parameteres are given in
Table \ref{tab:E1-V2-S3}.
\begin{table}[H]
    \centering
    \begin{tabular}{|c|c|c|c|c|c|c|c|c|c|}
        \hline
        $P_{D,k}(x)$ & $P$ & $\sigma_{\upsilon}$ & $\sigma_{\epsilon}$ & $T_H$ & $T_d$ & $T_p$ & $T_l$ & $T_{text}$ & $T_{bbox}$\\ \noalign{\hrule
        height 1.5pt}
        0.3 & $\diag(100,100,100,100)$ & 0.1 & 30 & 1 & 3 & 0.1 & 0.01 & 0.3 & 0.3\\
        \hline
    \end{tabular}
    \caption{The parameter settings for Experiment E1-V2-S3 with the dynamic detection probability.}
    \label{tab:E1-V2-S3}
\end{table}

Figure \ref{fig:E1-V2-S3} reveals a better object detection. As a result, the tracking of objects is more precise than
in the other settings.
\begin{itemize}
    \item \textbf{\ref{fig:E1-V2-S3:01}:} Starting with four tracked targets in the frame no. 83.
    \item \textbf{\ref{fig:E1-V2-S3:02}:} As in Experiment E1-V1, the motion and observation noise covariances
    seem to be too large for this scenario. The third and the fourth car's measurement reaches the validation region of
    each other. This situation creates new undesired targets.
    \item \textbf{\ref{fig:E1-V2-S3:03}:} The exact same situation happens in this frame.
    \item \textbf{\ref{fig:E1-V2-S3:04}:} Another car reaches the spawning point.
    \item \textbf{\ref{fig:E1-V2-S3:05}:} Five targets appear in the scene, all of them correctly tracked.
    \item \textbf{\ref{fig:E1-V2-S3:06}:} The scenario ends with four properly detected and tracked objects.
\end{itemize}

Even though the two cars driving side by side cause the filter with given parameters a few modest problems, the merging
and
the pruning steps can usually deal with such situations, as seen in Figure \ref{gr:E1-V2-S3}. The line
showing
the number of displayed targets almost copies the line showing the true counts. The graph also shows the problem of
two neighbouring targets and the problem of false detections. The peaks in the red line show these false detections,
the filter remains unaffected and holds the true number of targets.


As in Experiment \textit{E1-V1}, this setting outperform the other settings variations. Moreover, due to the more
precise
object detection, the filter is able to deal with problems such as two targets appearing in the same neighbourhood or
false
detections. However, settings \textit{S3} is very sensitive to motion and observation noise parametrization, thus
these covariance matrices have to be set carefully.

\begin{figure}[H]
    \centering
    \includegraphics[width=\linewidth]{../../../experiments/E1/V2/DINO/dino_det}
    \caption{Development chart of the number of detected targets, targets in the filter's queue, displayed targets and
    true targets' count.}
    \label{gr:E1-V2-S3}
\end{figure}

\begin{figure}[H]
    \centering
    \begin{subfigure}{0.48\textwidth}
        \centering
        \includegraphics[width=\linewidth]{../../../experiments/E1/V2/DINO/83}
        \caption{Frame number: 83.}
        \label{fig:E1-V2-S3:01}
    \end{subfigure}
    \begin{subfigure}{0.48\textwidth}
        \centering
        \includegraphics[width=\linewidth]{../../../experiments/E1/V2/DINO/89}
        \caption{Frame number: 89.}
        \label{fig:E1-V2-S3:02}
    \end{subfigure}
    \\
    \begin{subfigure}{0.48\textwidth}
        \centering
        \includegraphics[width=\linewidth]{../../../experiments/E1/V2/DINO/94}
        \caption{Frame number: 94.}
        \label{fig:E1-V2-S3:03}
    \end{subfigure}
    \begin{subfigure}{0.48\textwidth}
        \centering
        \includegraphics[width=\linewidth]{../../../experiments/E1/V2/DINO/98}
        \caption{Frame number: 98.}
        \label{fig:E1-V2-S3:04}
    \end{subfigure}
    \\
    \begin{subfigure}{0.48\textwidth}
        \centering
        \includegraphics[width=\linewidth]{../../../experiments/E1/V2/DINO/103}
        \caption{Frame number: 103.}
        \label{fig:E1-V2-S3:05}
    \end{subfigure}
    \begin{subfigure}{0.48\textwidth}
        \centering
        \includegraphics[width=\linewidth]{../../../experiments/E1/V2/DINO/109}
        \caption{Frame number: 109.}
        \label{fig:E1-V2-S3:06}
    \end{subfigure}
    \caption{Image sequence of tracked objects using the GM-PHD filter with the dynamic detection probability, the DINO
    object detector and the SAM image segmentation model.}
    \label{fig:E1-V2-S3}
\end{figure}






\section{E1: paper}
\section{E2: settings comparison}
\section{E3: add obstacle, compare with PHD}
\section{E4: comparison of mean vector}
\section{E5: comparison of covariance}
