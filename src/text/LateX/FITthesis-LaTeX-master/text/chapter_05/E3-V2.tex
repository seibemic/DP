\section{E3: Changing the model}
\renewcommand{\Ex}{E3}
Up to this point, all experiments have employed the observation and state-space models outlined in the introduction
of the Experiments section (see Section \ref{sec:experiments}). However, using these models in scenarios presented by videos \textit{V1, V2, V3} comes with several disadvantages. The state-space model assumes an uniform position uncertainty in all directions for targets. Additionally, the videos are not captured from a bird's-eye point of view, but from an angled perspective. This camera angle causes the detected targets to appear smaller when they are further away, and larger when they are closer to the camera. Consequently, this effect alters the actual movement model of the targets, which our current model does not account for.


Considering the videos utilized in this study, we can leverage prior knowledge regarding the movement patterns of the potential targets. It is established that cars are inclined to move in the
direction
of the road rather than backwards or sideways.
\subsection{V2}
\renewcommand{\Vs}{V2}
For this experiment, the video \textit{V2} with the added crossing road as an obstacle is used. The video is observed
at the rate 14 fps.

The state space CVM model is employed with the transition matrix

\begin{align}
    F_k &=
    \begin{bmatrix}
        1 & 0 & 2\Delta & 0\\
        0 & 1 & 0 & \Delta \\
        0 & 0 & 1.1 & 0 \\
        0 & 0 & 0 & 1.1
    \end{bmatrix},
\end{align}
where $\Delta = 1$,
and process covariance matrix
\begin{align}
    Q_k &=
    \begin{bmatrix}
        2 & 0.5 & 0 & 0\\
        0.5 & 1 & 0 & 0 \\
        0 & 0 & 1 & 0 \\
        0 & 0 & 0 & 1
    \end{bmatrix}
    \cdot 0.003.
    \label{eq:exp_E3-V2_Q}
\end{align}

The observation covariance matrix is reformulated as
\begin{align}
    R_k &=
    \begin{bmatrix}
        2 & 0.5 \\
        0.5 & 1
    \end{bmatrix}
    \cdot 20.
    \label{eq:exp_E3-V2_R}
\end{align}

The spawning point providing nearby targets with initial covariance matrix $P$ is established as
\begin{align}
    P_k &=
    \begin{bmatrix}
        80 & 20 & 0 & 0 \\
        20 & 40 & 0 & 0 \\
        0 & 0 & 40 & 0 \\
        0 & 0 & 0 & 40
    \end{bmatrix}. \label{eq:exp_E3-V2_P}
\end{align}

These model formulations ensure that the target's covariance takes an ellipsoidal shape rather than a circular one. These ellipses are oriented in the direction of the road, indicating that uncertainty increases more in alignment with the road's direction. This implies a lower probability of a car being positioned next to the road in the subsequent time step.
\subsection{V2b -- GM-PHD with the dynamic detection probability}
The GM-PHD filter with the dynamic detection probability is analysed only in this experiment. The best performing settings are used - \textit{S3}, i.e., grounded SAM.

\subsubsection{S3 -- Grounded SAM}
\renewcommand{\Set}{S3}
In Table \ref{tab:\Ex-\Vs-\Set}, note that the standard pruning threshold $T_p$ is the same as the lowered pruning
threshold $T_l$. Due to the better model settings and the dynamic detection probability, it is not neccesarry to lower the pruning threshold for targets in \textit{detected} and \textit{hidden} state.
\begin{table}[H]
    \centering
    \begin{tabular}{|c|c|c|c|c|c|c|c|c|c|}
        \hline
        $P_{D,k}(x)$ & $P$ & $\sigma_{\upsilon}$ & $\sigma_{\epsilon}$ & $T_H$ & $T_d$ & $T_p$ & $T_l$ & $T_{text}$ & $T_{bbox}$\\ \noalign{\hrule
        height 1.5pt}
        0.3 & see Eq. \ref{eq:exp_E3-V2_P} & see Eq. \ref{eq:exp_E3-V2_Q} & see Eq. \ref{eq:exp_E3-V2_R} & 0.6 & 3 & 0.1 & 0.1 & 0.3 & 0.3\\
        \hline
    \end{tabular}
    \caption{The parameter settings for Experiment {\Ex-\Vs-\Set} with the dynamic detection probability.}
    \label{tab:\Ex-\Vs-\Set}
\end{table}

The adjusted models project in Figures \ref{fig:\Ex-\Vs-\Set_1} and \ref{fig:\Ex-\Vs-\Set_2}. The spawning point, as
well as the covariances given by the covariance matrix $P$, has
an ellipsoidal shape showing the uncertainty. Two cars
are observed in the sequence. Let us call the car driving in the right highway line in the direction of travel $C_R$
and the left car as $C_L$.

\begin{itemize}
    \item \textbf{\ref{fig:\Ex-\Vs-\Set:01}:} Two targets approach the obstacle. Both cars are successfully detected and initialized.
    \item \textbf{\ref{fig:\Ex-\Vs-\Set:02}:} Target $C_R$ is detected for the final time before reaching the obstacle.
    \item \textbf{\ref{fig:\Ex-\Vs-\Set:03}:} The object detector still manages to detect $C_L$, but its position is now beneath the obstacle, resulting in both targets being classified as \textit{hidden}.
    \item \textbf{\ref{fig:\Ex-\Vs-\Set:04}:} Both targets remain hidden, showcasing the effect of the altered models. Notably, the covariances are elongated in the direction of travel and do not extend beyond the traffic line.
    \item \textbf{\ref{fig:\Ex-\Vs-\Set:05}:} Despite minimal fine-tuning of the model parameters for this experiment, the predicted mean positions are nearly accurate. Both cars are detected once more, allowing the filter to obtain measurements.
    \item \textbf{\ref{fig:\Ex-\Vs-\Set:06}:} The additional false target is removed and both targets are tracked with small ellipsoidal covariances.
\end{itemize}


\begin{figure}[H]
    \centering
    \begin{subfigure}{\textwidth}
        \centering
        \includegraphics[width=0.85\linewidth]{../../../experiments/\Ex/\Vs/DINO/44}
        \caption{Frame number: 44.}
        \label{fig:\Ex-\Vs-\Set:01}
    \end{subfigure}
    \begin{subfigure}{\textwidth}
        \centering
        \includegraphics[width=0.85\linewidth]{../../../experiments/\Ex/\Vs/DINO/48}
        \caption{Frame number: 48.}
        \label{fig:\Ex-\Vs-\Set:02}
    \end{subfigure}
    \\
    \begin{subfigure}{\textwidth}
        \centering
        \includegraphics[width=0.85\linewidth]{../../../experiments/\Ex/\Vs/DINO/50}
        \caption{Frame number: 50.}
        \label{fig:\Ex-\Vs-\Set:03}
    \end{subfigure}
    \caption{Image sequence of tracked objects using the GM-PHD filter with the dynamic detection probability, Grounded SAM and adjusted models -- part 1.}
    \label{fig:\Ex-\Vs-\Set_1}
\end{figure}
\begin{figure}[H]

    \begin{subfigure}{\textwidth}
        \centering
        \includegraphics[width=0.85\linewidth]{../../../experiments/\Ex/\Vs/DINO/53}
        \caption{Frame number: 53.}
        \label{fig:\Ex-\Vs-\Set:04}
    \end{subfigure}
    \\
    \begin{subfigure}{\textwidth}
        \centering
        \includegraphics[width=0.85\linewidth]{../../../experiments/\Ex/\Vs/DINO/54}
        \caption{Frame number: 54.}
        \label{fig:\Ex-\Vs-\Set:05}
    \end{subfigure}
    \begin{subfigure}{\textwidth}
        \centering
        \includegraphics[width=0.85\linewidth]{../../../experiments/\Ex/\Vs/DINO/58}
        \caption{Frame number: 58.}
        \label{fig:\Ex-\Vs-\Set:06}
    \end{subfigure}
    \caption{Image sequence of tracked objects using the GM-PHD filter with the dynamic detection probability, Grounded SAM and adjusted models -- part 2.}
    \label{fig:\Ex-\Vs-\Set_2}
\end{figure}


Table \ref{tab:\Ex-\Vs-\Set_pd_w} and Figure \ref{gr:\Ex-\Vs-\Set} display the values of the detection probabilities
of targets $
C_L$
and $C_R$ in
corresponding
frames. If the targets are visible in the scene, the detection probability is large. Once the targets are hidden
behind the obstacle, the detection probability rapidly drops. The value of the probability is about 0.1 from frame $
50$ to $54$. The effect of this low detection probability is projected in Table \ref{tab:\Ex-\Vs-\Set_pd_w} and Graph \ref{gr:\Ex-\Vs-\Set} as well. The targets'
weights are slowly decreasing, sustaining the targets' existence in period without any measurement.
\begin{table}[H]
    \centering
    \begin{tabular}{|c|c|c|c|c|}
        \hline
        & \multicolumn{2}{|c|}{Detection probabilities} & \multicolumn{2}{|c|}{Targets' weights} \\ \noalign{\hrule height 1.5pt}
        Frame number & $p_{D,k}(C_L)$ & $p_{D,k}(C_{R})$ & $w_{k}(C_L)$ & $w_{k}(C_R)$\\ \noalign{\hrule height 1.5pt}
        44 & 0.934 & 0.861 & 1 & 1\\
        \hline
        45 & 0.920 & 0.881 & 1 & 1\\
        \hline
        46 & 0.891 & 0.908 & 1 & 1\\
        \hline
        47 & 0.900 & 0.802 & 1 & 1\\
        \hline
        48 & 0.879 & 0.728 & 1 & 1\\
        \hline
        49 & 0.677 & 0.534 & 1 & 1\\
        \hline
        50 & 0.457 & 0.114 & 1 & 0.876\\
        \hline
        51 & 0.135 & 0.129 & 0.855 & 0.786\\
        \hline
        52 & 0.138 & 0.088 & 0.730 & 0.710\\
        \hline
        53 & 0.142 & 0.098 & 0.619 & 0.634\\
        \hline
        54 & 0.132 & 0.294 & 0.532 & 0.823\\
        \hline
        55 & 0.321 & 0.422 & 1 & 1\\
        \hline
        56 & 0.865 & 0.898 & 1 & 1\\
        \hline
        57 & 0.934 & 0.922 & 1 & 1\\
        \hline
        58 & 0.941 & 0.924& 1 & 1\\
        \hline
    \end{tabular}
    \caption{The evolution of detection probabilities and weights of targets $C_L$ and $C_R$.}
    \label{tab:\Ex-\Vs-\Set_pd_w}
\end{table}

\begin{figure}[H]
    \centering
    \includegraphics[width=1\linewidth]{../../../experiments/\Ex/\Vs/DINO/pd_w_evolution}
    \caption{The evolution of targets' detection probabilities and weights.}
    \label{gr:\Ex-\Vs-\Set}
\end{figure}

This experiment illustrates that even in scenarios with challenging conditions, such as angled recordings, it is
possible to configure the model using prior scene knowledge in a manner that enables the filter to effectively handle
obstacles obstructing the camera view.
Moreover, these model settings are capable of addressing the necessity for dynamically setting the noise covariances.
Table \ref{tab:\Ex-\Vs-\Set_pd_w} and Graph \ref{gr:\Ex-\Vs-\Set} demonstrate the effect of the dynamic detection probability
on targets' weights.
Note that targets
in Figures \ref{fig:\Ex-\Vs-\Set_1} and \ref{fig:\Ex-\Vs-\Set_2} are correctly in \textit{hidden} state. Thus, if the
obstacle would be larger, due to possibility of changing the pruning threshold for targets in \textit{detected} and \textit{hidden} state, the targets might survive even longer. This additional threshold possibility does not allow the creation of other false targets that would appear if the standard pruning threshold was lowered.