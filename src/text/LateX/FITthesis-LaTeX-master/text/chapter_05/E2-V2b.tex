\subsection{V2b}
\renewcommand{\Vs}{V2b}
The preceding experiment evaluates the performance of the GM-PHD filter with dynamic detection probability under settings \textit{S1} on a video featuring an added obstacle with color characteristics similar to the surrounding scene. In this experiment, the obstacle possesses slightly different color characteristics. The primary objective of this experiment is to assess the efficacy of the pruning step influenced by the Markov process.

\subsection{V2b -- GM-PHD with dynamic detection probability}
The conditions of this experiment are kept the same is in Experiment \ref{sec:E2-V2a}, i.e., the YOLO object detector and segmentation model is used.

\subsubsection{S1 -- YOLO + YOLO}
\renewcommand{\Set}{S1}
For fair comparison, the parameter settings included in Table \ref{tab:\Ex-\Vs-\Set} are left the same.
\begin{table}[H]
    \centering
    \begin{tabular}{|c|c|c|c|c|c|c|c|c|}
        \hline
        $P_{D,k}(x)$ & $P$ & $\sigma_{\upsilon}$ & $\sigma_{\epsilon}$ & $T_H$ & $T_d$ & $T_p$ & $T_l$ & $T_{YOLO}$ \\ \noalign{\hrule
        height 1.5pt}
        0.3 & $diag(40,40,40,40)$ & 0.04 & 120 & 0.5 & 3 & 0.1 & 0.001 & 0.3\\
        \hline
    \end{tabular}
    \caption{The parameter settings for experiment {\Ex-\Vs-\Set} with dynamic detection probability.}
    \label{tab:\Ex-\Vs-\Set}
\end{table}

In Figure \ref{fig:\Ex-\Vs-\Set} we can see the tracking performance of the GM-PHD filter with settings \textit{S1} on traffic situation with an added obstacle.
\begin{itemize}
    \item \textbf{\ref{fig:\Ex-\Vs-\Set:01}:} This sequence starts with frame no. 44. Due to the targets' close postions, more than 2 targets appears in the place where only two true targets are present.
    \item \textbf{\ref{fig:\Ex-\Vs-\Set:02}:} The targets enter under the obstacle.
    \item \textbf{\ref{fig:\Ex-\Vs-\Set:03}:} The first frame with misdetected objects. The targets are already in \textit{hidden state}.
    \item \textbf{\ref{fig:\Ex-\Vs-\Set:04}:} The targets' predicted positions are already behind the true targets' positions.
    \item \textbf{\ref{fig:\Ex-\Vs-\Set:05}:} As the predicted covariance grows, the targets merged into a single one. The cars are clearly seen, but the YOLO model does not detect them.
    \item \textbf{\ref{fig:\Ex-\Vs-\Set:06}:} The cars are detected and initialized by the target. The predicted black bounding box still interfers with the added obstacle, thus the target is considered as hidden.
    \item \textbf{\ref{fig:\Ex-\Vs-\Set:07}:} In this frame, the predicted black bounding box moves to the area with natural traffic line. The false target is in \textit{dead} state and is removed immediately.
    \item \textbf{\ref{fig:\Ex-\Vs-\Set:08}:} The targets continue in their paths with correct positions.
\end{itemize}

In this experiment, we have demonstrated, that if an obstacle differs from the targets' background scene, the pruning given by the Markov process works exceptionally well. The downside of this approaches lies in additional settings of further parameters.
\begin{figure}[H]
    \centering
    \begin{subfigure}{0.48\textwidth}
        \centering
        \includegraphics[width=\linewidth]{../../../experiments/\Ex/\Vs/YOLO/44}
        \caption{Frame number: 44.}
        \label{fig:\Ex-\Vs-\Set:01}
    \end{subfigure}
    \begin{subfigure}{0.48\textwidth}
        \centering
        \includegraphics[width=\linewidth]{../../../experiments/\Ex/\Vs/YOLO/47}
        \caption{Frame number: 47.}
        \label{fig:\Ex-\Vs-\Set:02}
    \end{subfigure}
    \\
    \begin{subfigure}{0.48\textwidth}
        \centering
        \includegraphics[width=\linewidth]{../../../experiments/\Ex/\Vs/YOLO/50}
        \caption{Frame number: 50.}
        \label{fig:\Ex-\Vs-\Set:03}
    \end{subfigure}
    \begin{subfigure}{0.48\textwidth}
        \centering
        \includegraphics[width=\linewidth]{../../../experiments/\Ex/\Vs/YOLO/54}
        \caption{Frame number: 54.}
        \label{fig:\Ex-\Vs-\Set:04}
    \end{subfigure}
    \\
    \begin{subfigure}{0.48\textwidth}
        \centering
        \includegraphics[width=\linewidth]{../../../experiments/\Ex/\Vs/YOLO/56}
        \caption{Frame number: 56.}
        \label{fig:\Ex-\Vs-\Set:05}
    \end{subfigure}
    \begin{subfigure}{0.48\textwidth}
        \centering
        \includegraphics[width=\linewidth]{../../../experiments/\Ex/\Vs/YOLO/57}
        \caption{Frame number: 57.}
        \label{fig:\Ex-\Vs-\Set:06}
    \end{subfigure}
    \\
    \begin{subfigure}{0.48\textwidth}
        \centering
        \includegraphics[width=\linewidth]{../../../experiments/\Ex/\Vs/YOLO/58}
        \caption{Frame number: 58.}
        \label{fig:\Ex-\Vs-\Set:07}
    \end{subfigure}
    \begin{subfigure}{0.48\textwidth}
        \centering
        \includegraphics[width=\linewidth]{../../../experiments/\Ex/\Vs/YOLO/59}
        \caption{Frame number: 59.}
        \label{fig:\Ex-\Vs-\Set:08}
    \end{subfigure}
    \caption{Image sequence of tracked objects using GM-PHD filter with dynamic detection probability and YOLO only.}
    \label{fig:\Ex-\Vs-\Set}
\end{figure}

