\chapter{Dynamic time and state varying detection probability}
In this work, we focus on tracking objects in frame sequence using target tracking methods and object detection and
segmentation algorithms. Object detectors were discussed in previous section, as they serve as a sensor for getting
measurements from image.

Note, that in all target tracking, especially multi-target tracking algorithms for cluttered environments and targets
birth and dead possibilities, there is a parameter called detection probability ($p_D$). This parameter, as it name
suggest, is a probability, that the target is detected in particular time. It is also commonly used as a constatnt
time and state independent variable. As we want to improve the performance of multi-target algorithms in video data
in scenarios, where the targets may be hidden by an obstacle or just reduce impacts of a misdetection of a sensor, it
is appropriate to have a dynamic time- and state-varying detection probability.
\section{Problem definition}
Even though it is possible to fine tune YOLO model on own datasets to either improve performance in detection of
pretrained classes, or to train the model for detecting other classes, with over 70 pretrained class instances it was
not neccesary to fine tune own YOLO model in this work. The model also detects all class instances, it knows and
appear in a scene. For testing purposes we filter only objects, we want to detect and get measurements from.
\begin{figure}
  \caption{All classes, filtered classes}
\end{figure}

As a multi-target tracking algorithm, we chose Probabilist Hypothesis density filter, as it is on simpler side of RFS
-based methods and is computationally very effective in comparison with filters such as the CPHD or the PMBM filter,
which may seem
to be more appropriate for this task. YOLO is a very effective, real-time object detector and the PHD filter is the
only one RFS filter that even though is not real-time, is the closest one to be.

To modify the PHD filter for our use case, first we need to define a few assumptions that are very similar to
assumptions of PHD filter and GM-PHD filter in Section \ref{sec:phdfilter} and \ref{sec:gmphdFilter}.
\begin{enumerate}
  \item Each target evolves and generates observations independently of each other.
  \item Clutter is Poisson and independent of target-originated measurements.
  \item The predicted multi-target RFS governed by $p_{k|k-1}$ is Poisson.
  \item Each target follows a linear Gaussian dynamical model and the sensor has a linear
  Gaussian measurement model, i.e.,
        \begin{align}
          f_{k|k-1}(x|\zeta) &= \mathcal{N}(x; F_{k-1}\zeta, Q_{k-1}),\\
          g_k(z|x) &= \mathcal{N}(z;H_kx, R_k),
        \end{align}
  where $\mathcal{N}(\cdot;m,P)$ is a Gaussian density with mean $m$ and covariance $P$, $F_{k-1}$ is the state transition matrix, $Q_{k-1}$ is the process noise covariance, $H_k$ is the observation matrix and $R_k$ is the observation noise covariance.
  \item The survival and detection probabilities are state independent, i.e.,
  \begin{align}
    p_{S,k}(x) &= p_{S,k},\\
    p_{D,k}(x) &= p_{D,k}.
  \end{align}
  \item The intesities of the birth and spawn RFSs are Gaussian mixtures of the form
  \begin{align}
    \gamma_k(x) &= \sum_{i=1}^{J_{\gamma,k}}w_{\gamma,k}^{(i)} \mathcal{N}(x; m_{\gamma.k}^{(i)}, P_{\gamma,k}^{(i)}),\\
    \beta_{k|k-1}(x|\zeta) &= \sum_{j=1}^{J_{\beta,k}} w_{\beta,k}^{(j)} \mathcal{N}(x;F_{\beta,k-1}^{(j)}\zeta + d_{\beta,k-1}^{(j)}, Q_{\beta,k-1}^{(j)}),
  \end{align}
\end{enumerate}

Note, that assumptions 1,2 are very strong. 1 - may generate more and depedently. 2 - clutter may be dependent. 5 -
divide into two parts. 6 - remove beta
\section{Dynamic detection probability in video data}
\section{Modified pruning for GM-PHD filter}
\section{Merging in GM-PHD filter with dynamic detection probability}